%%%some text about buffers
\section{Buffer objects}
Buffers play an important role within the library. Usually you do not need 
to care about them. However, for library developers it is important 
to know how they work.

A buffer represents a number of elements of a particular size linearly 
stored in memory. Thus is behaves somehow like an array or one of the 
container templates provided by the STL. However, it follows its own 
allocation and reallocation policy which will be described here in more detail.  

The aim of a buffer is to allocate memory for a number of objects of equal 
size. Therefore, the amount of memory allocated depends on two parameters
\begin{enumerate}
  \item the {\em element size } which is the size of each element in bytes
  \item the {\em size } of the buffer which is the number of elements in the
  buffer.
\end{enumerate}
The total amount of memory which must be allocated to hold this data is 
than given with
\begin{align}
n_{\mathrm{tot}} = \mathrm{element size}\times\mathrm{size}.
\end{align}

All buffers should be derived from the {\tt BufferObject} base class which 
holds the minimal interface each buffer object should implement. 
The base class itself only provides simple managment functions for the
size of the allocated memory. However, it already implies some of the 
policies that each buffer object should follow. 
There are three situations a buffer object must face
\begin{enumerate}
  \item instantiation
  \item allocation
  \item assignment
  \item reallocation
\end{enumerate} 

\subsection{Instantiation of buffer objects}

Buffers allow the allocation of memory to be posponed to a later point in 
time after their creation. This of coarse has some implications. 
It is necessary to keep track of a buffers state in every method that 
can effect the allocated memory. 




The template {\tt Buffer<T>} is a concrete implementation 
